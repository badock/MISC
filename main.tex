\documentclass{article}         %% ceci est un commentaire (apres le caractere %)
    %% adapte le style article aux conventions francophones
\usepackage[T1]{fontenc}          %% permet d'utiliser les caractères accentués
% \usepackage[dvips]{graphicx}      %% permet d'importer des graphiques au format .EPS (postscript)
\usepackage{fancybox}		   %% package utiliser pour avoir un encadré 3D des images
\usepackage{makeidx}              %% permet de générer un index automatiquement
\usepackage{float}
\usepackage{todonotes}

\usepackage[toc,page]{appendix} 

% BEGIN: add footnote in tabular

\newcounter{notetab}
\newcommand{\zero}{\setcounter{notetab}{0}}
\newcommand{\ntm}{\footnotemark\addtocounter{notetab}{1}}
\newcommand{\initnt}{\addtocounter{footnote}{-\value{notetab}}}
\newcommand{\ntt}[1]{%
   \addtocounter{footnote}{1}
   \footnotetext{#1}}

% END: add footnote in tabular

% BEGIN: remove acm copyright
\usepackage{etoolbox}
\makeatletter
\patchcmd{\maketitle}{\@copyrightspace}{}{}{}
\makeatother
% END: remove acm copyright

\title{Introduction to Play! framework}     %% \title est une macro, entre { } figure son premier argument
\author{Jonathan Pastor}        %% idem

\makeindex		    %% macro qui permet de générer l'index
\bibliographystyle{prsty}	  %% le style utilisé pour créer la bibliographie
\begin{document}                  %% signale le début du document

\maketitle                        %% produire à cet endroit le titre de l'article à partir des informations fournies ci-dessus (title, author)


\section{Introduction}

The objective of this course is to introduce the basic features of a modern web framework, through the study of a java compatible Model-View-Controller \footnote{cf. http://en.wikipedia.org/wiki/Model-view-controller} (MVC) web framework: Play! \footnote{cf. http://www.playframework.com/documentation/1.2.5/home}.
\newline
\newline
During this course, the student will learn how to develop a small web application. This application will use a database for storing objects, and users will interact with it through using web forms.
\newline
\newline
Students will be asked to work in pairs: each pair will develop a dynamic website: the specifications are given in the section \ref{specs}. The specifications are a set of constraints: Each constraint that is satisfied gives points, These constraints will be evaluated by a jury during the oral defence of the project.



\section{The Subject of the project}

The pairs are free to find a subject that will drive their project. Students are free to use this example of a website that sells sandwiches: 

\begin{quotation}
A sandwich shop asks Students of Ecole des Mines de Nantes to develop an online shop. Customers will be able to order sandwich via a front-end website. When an order is validated by a customer, it is processed by an administrator that will prepare the order. The administrator can create or modify sandwiches. \par
\raggedleft Project subject for year 2012-2013.
\end{quotation}

\begin{quotation}
A customer will create orders. Orders belong to one customer. An order is a set of association between a sandwich and a number. A sandwich is composed by a set of ingredients. \par
\raggedleft Example of data structure for year 2012-2013.
\end{quotation}



\section{The Specifications}
\label{specs}

\zero
\begin{tabular}{|p{2cm}|p{10cm}|p{1cm}|}
  \hline
  Constraint & Description & Points \\

  \hline
  Controller-1 & Develop a controller for the frontend & 1 pts \\
  Controller-2 & Develop a controller for the backend & 1 pts \\
  Controller-3 & At least one controller asks data with a custom SQL query & 1 pts \\

  \hline
  View-1 & Display data (a list) with an HTML table & 1 pts \\
  View-2 & Display data (a list) with an HTML list & 1 pts \\
  View-3 & Display data (an element) with details & 1 pts \\
  View-4 & Create a transition between 2 views & 1 pts \\

  \hline
  Database-1 & Database contains at least one complex entity (that contains a set of other entities, cf appendix \ref{complex_entity}) & 3 pts \\
  Database-1 & The database is in third normal form \ntm & 1 pts \\

  \hline
  CRUD & At least one complex data entity have create, read, update and delete actions. These actions are located in the backend. & 3 pts \\

  \hline
  AJAX-1 & At least one controller have a method that produce JSON or XML & 1 pts \\
  AJAX-2 & At least one view gets data dynamically with an AJAX request & 1 pts \\

  \hline
  Design-1 & Use a CSS framework (Bootstrap\ntm, Zurb Foundation \ntm, PureCSS\ntm, ...) & 1 pts \\
  Design-2 & The application has a clean design & 1 pts \\
  Design-2 & The application has a good ergonomy & 1 pts \\
  Design-3 & The application has been developed around prototyping methodology \ntm & 1 pts \\

  \hline
  Total & \multicolumn{2}{r|}{   20 pts \hspace{1 mm}} \\

  \hline
  Bonus-1 & Display statistics with a framework like D3.js \ntm & 1 pts \\

  \hline
\end{tabular}

\initnt
\ntt{cf. http://en.wikipedia.org/wiki/Third\_normal\_form}
\ntt{cf. http://getbootstrap.com/}
\ntt{cf. http://foundation.zurb.com/}
\ntt{cf. http://purecss.io/}
\ntt{cf. http://en.wikipedia.org/wiki/Software\_prototyping}
\ntt{cf. http://d3js.org/}

\section{Evaluation}

During the oral defence of the project, each constraint will be evaluated. The jury will check if the students have a good comprehension of the technologies seen during this course. The sum of the points given by the satisfaction of the constraints will give an idea of the final mark.

\newpage

\begin{appendices}

\section{Complex Data Entity}
\label{complex_entity}

Here is an example of complex data (entity Sandwich):

\begin{scriptsize}
\begin{verbatim}
class Sandwich {
    String name;
    List<Ingredient> ingredients;
}

[...]

class Ingredient {
    String name
}
\end{verbatim}
\end{scriptsize}

In this example a sandwich contains a set of ingredients.
\end{appendices}


\end{document} 
