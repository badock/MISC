\documentclass{article}         %% ceci est un commentaire (apres le caractere %)
    %% adapte le style article aux conventions francophones
\usepackage[T1]{fontenc}          %% permet d'utiliser les caractères accentués
% \usepackage[dvips]{graphicx}      %% permet d'importer des graphiques au format .EPS (postscript)
\usepackage{fancybox}		   %% package utiliser pour avoir un encadré 3D des images
\usepackage{makeidx}              %% permet de générer un index automatiquement
\usepackage{float}
\usepackage{todonotes}

\usepackage{amsmath}
\usepackage[toc,page]{appendix} 

% BEGIN: add footnote in tabular

\newcounter{notetab}
\newcommand{\zero}{\setcounter{notetab}{0}}
\newcommand{\ntm}{\footnotemark\addtocounter{notetab}{1}}
\newcommand{\initnt}{\addtocounter{footnote}{-\value{notetab}}}
\newcommand{\ntt}[1]{%
   \addtocounter{footnote}{1}
   \footnotetext{#1}}

% END: add footnote in tabular

% BEGIN: remove acm copyright
\usepackage{etoolbox}
\makeatletter
\patchcmd{\maketitle}{\@copyrightspace}{}{}{}
\makeatother
% END: remove acm copyright

\title{Introduction to Play! framework}     %% \title est une macro, entre { } figure son premier argument
\author{Jonathan Pastor}        %% idem

\makeindex		    %% macro qui permet de générer l'index
\bibliographystyle{prsty}	  %% le style utilisé pour créer la bibliographie
\begin{document}                  %% signale le début du document

\maketitle                        %% produire à cet endroit le titre de l'article à partir des informations fournies ci-dessus (title, author)


\section{Introduction}

The objective of this course is to introduce the basic features of a modern web framework, through the study of a java compatible Model-View-Controller \footnote{cf. http://en.wikipedia.org/wiki/Model-view-controller} (MVC) web framework: Play! \footnote{cf. http://www.playframework.com/documentation/1.2.5/home}.
\newline
\newline
During this course, students will learn how to develop a small web application. This application will use a database for storing objects, and users will interact with it by using web forms.
\newline
\newline
Students will be asked to work in pairs: each pair will develop a dynamic website. Specifications are given in the section \ref{specs} : it is a set of constraints where each satisfied constraint gives points. The satisfaction of the constraints will be evaluated during an oral defense.



\section{The Subject of the project}

The pairs will develop an application that will be the response to the following needs:

\begin{quotation}
Professors of the Computer Science (CS) department would like to be able to create meetings involving several participants. The involved people would review the meeting and let a comment about the proposed meeting. It would be also possible to handle conflicts between meetings. People may be notified when they take part to a meeting.\par
\raggedleft Project subject for year 2013-2014.
\end{quotation}

\begin{quotation}
Meetings are created by a user. A meeting involve one or several people. A meeting can contains comments from any member of the meeting. A member can receive notifications about some of his meetings.\par
\raggedleft Example of data structure for year 2013-2014.
\end{quotation}



\section{The Specifications}
\label{specs}

\subsection{Fonctional Specification (10pts)}

\zero
\begin{tabular}{|p{2cm}|p{10cm}|p{1cm}|}
  \hline
  Constraint & Description & Points \\

  \hline
  Fonctional-1 & An user can create, delete, modify and view a meeting. & 2 pts \\
  Fonctional-2 & Participants of a meeting can view their meetings. & 2 pts \\
  Fonctional-3 & Participants of a meeting can leave a comment on the meeting. & 2 pts \\
  Fonctional-4 & New meetings appears automagically without reloading the page (AJAX). & 2 pts \\
  Fonctional-5 & Implement a feature that improves the application & 2 pts \\
  \hline
  Total & \multicolumn{2}{r|}{   10 pts \hspace{1 mm}} \\
  \hline
  Bonus & The application is usable by members of the C.S. department. & 3 pts \\

  \hline
\end{tabular}


\subsection{Technical Specification (20pts)}

\zero
\begin{tabular}{|p{2cm}|p{10cm}|p{1cm}|}
  \hline
  Constraint & Description & Points \\

  \hline
  Controller-1 & Develop a controller for the frontend & 1 pts \\
  Controller-2 & Develop a controller for the backend & 1 pts \\
  Controller-3 & At least one controller asks data with a custom SQL query & 1 pts \\

  \hline
  View-1 & Display data (a list) with HTML & 2 pts \\
  View-2 & Display data (an element) with details & 1 pts \\
  View-3 & Create a transition between 2 views & 1 pts \\

  \hline
  Database-1 & Database contains at least one complex entity (that contains a set of other entities, cf appendix \ref{complex_entity}) & 3 pts \\
  Database-1 & The database is in third normal form \ntm & 1 pts \\

  \hline
  CRUD & At least one complex data entity have create, read, update and delete actions. These actions are located in the backend. & 3 pts \\

  \hline
  AJAX-1 & At least one controller have a method that produce JSON or XML. & 1 pts \\
  AJAX-2 & At least one view gets data dynamically with an AJAX request. & 1 pts \\

  \hline
  Design-1 & Use a CSS framework (Bootstrap\ntm, Zurb Foundation \ntm, PureCSS\ntm, ...). & 1 pts \\
  Design-2 & The application is simple and friendly to use & 2 pts \\
  Design-3 & The application has been developed around prototyping methodology. \ntm & 1 pts \\

  \hline
  Total & \multicolumn{2}{r|}{   20 pts \hspace{1 mm}} \\

  \hline
  Bonus-1 & Integrate the application with members given by LDAP. & 2 pts \\
  \hline
\end{tabular}

\initnt
\ntt{cf. http://en.wikipedia.org/wiki/Third\_normal\_form}
\ntt{cf. http://getbootstrap.com/}
\ntt{cf. http://foundation.zurb.com/}
\ntt{cf. http://purecss.io/}
\ntt{cf. http://en.wikipedia.org/wiki/Software\_prototyping}


\section{Evaluation}

Evaluation will be made during an oral defense : a jury will check if the students have a good comprehension of the technologies seen during this course; each constraint will be evaluated. The jury will then decide of a mark for the oral defense (10 pts).

\[\text{finalMark} = \frac{\text{technicalMark} + \text{functionalMark} + \text{oralMark}}{2}\]


\begin{appendices}

\section{Complex Data Entity}
\label{complex_entity}

Here is an example of complex data:

\begin{scriptsize}
\begin{verbatim}
class A {
    List<B> listOfBs;
}

[...]

class B {
}
\end{verbatim}
\end{scriptsize}

In this example, entity A can contains several B entities.
\end{appendices}


\end{document} 
