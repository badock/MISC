\documentclass{article}         %% ceci est un commentaire (apres le caractere %)
    %% adapte le style article aux conventions francophones
\usepackage[T1]{fontenc}          %% permet d'utiliser les caractères accentués
% \usepackage[dvips]{graphicx}      %% permet d'importer des graphiques au format .EPS (postscript)
\usepackage{fancybox}      %% package utiliser pour avoir un encadré 3D des images
\usepackage{makeidx}              %% permet de générer un index automatiquement
\usepackage{float}
\usepackage{todonotes}

\usepackage{amsmath}
\usepackage[toc,page]{appendix} 

% BEGIN: add footnote in tabular

\newcounter{notetab}
\newcommand{\zero}{\setcounter{notetab}{0}}
\newcommand{\ntm}{\footnotemark\addtocounter{notetab}{1}}
\newcommand{\initnt}{\addtocounter{footnote}{-\value{notetab}}}
\newcommand{\ntt}[1]{%
   \addtocounter{footnote}{1}
   \footnotetext{#1}}

% END: add footnote in tabular

% BEGIN: remove acm copyright
\usepackage{etoolbox}
\makeatletter
\patchcmd{\maketitle}{\@copyrightspace}{}{}{}
\makeatother
% END: remove acm copyright

\title{Introduction to Web development with the Django framework.}
\author{Walid Benghabrit, Florent Marchand de Kerchove, Jonathan Pastor}

\makeindex
\bibliographystyle{prsty}
\begin{document}

\maketitle


\section{Introduction}

The objective of this course is to introduce the basics of a modern web 
framework, through the study of a java compatible Model-View-Controller 
\footnote{cf. http://en.wikipedia.org/wiki/Model-view-controller} (MVC) web 
framework: Django. 
\footnote{https://www.djangoproject.com}.
\newline
\newline
During this course, students will learn how to develop a small web application. 
This application will use a database for storing objects, and users will 
interact with it by using web forms.
\newline
\newline
Students will be asked to work in pairs: each pair will develop a dynamic 
website. Specifications are given in the section \ref{specs} : it is a set of 
constraints where each satisfied constraint gives points. The satisfaction of 
the constraints will be evaluated during an oral defense.


\section{Objective of the course}

Students will learn how to develop modern web applications. The example used in
this course will be the development of a web application that will be in charge
of managing scores, as during sportive events. 


\section{The Subject of the project}

Students will be divided in pairs: each pairs will develop an application that 
will be responsible of managing scores: : users of this application will be able
in a first time to insert/modify results of confrontation between players, and 
in a second time to provide advanced features such as different kind of sportive
events (tournaments, championships, leagues, ...) or analysis of results.

\begin{quotation}
We, members of the Computer Science (CS) department, we think that a healthy 
spirit in a healthy body is a good leimotiv. That is why we enjoy doing sportive
activities, especially when they require a competitive spirit. Unfortunately, 
due to our many sportive activities, we have bad memories and we cannot remember
the scores. That is why we would like to be able to manage the results of these
confrontations with beautiful web application. At the end of each confrontations
between members of the CS department, they will insert a new entry in the 
database of this web application, without requiring a big effort.\par 
\raggedleft Project subject for year 2014-2015.
\end{quotation}

\begin{quotation}
Scores are created by one of the particpants of the event (a user). An event
can involve one or several people. A confrontation result can contains comments 
from any of the participants of the meeting. When a new score is added, its
participants may be notified.\par \raggedleft Minimal web application for year 
2014-2015.
\end{quotation}

The previous illustrates the minimal requirements for this project. \textbf{
Students are asked to develop at least one original feature that will complete 
this web application}, such as providing:

\begin{itemize}
  \item support for diffent kind of confrontations (tournament, championship, 
  leagues, ....).
  \item support for advanced analysis of scores.
  \item implementig a complex data model.
\end{itemize}

\section{The Specifications}
\label{specs}

\subsection{Fonctional Specification (10pts)}

\zero
\begin{tabular}{|p{2cm}|p{10cm}|} %p{1cm}|}
  \hline
  Constraint & Description\\ % & Points \\

  \hline
  Fonctional-1 & An user can create, delete, modify and view a contest. \\% & 2 pts \\
  Fonctional-2 & Player can view his contests. \\%  & 2 pts \\
  Fonctional-3 & Participants of a contest can leave a comment on the contest. \\% & 2 pts \\
  Fonctional-4 & New contest appears automatically without reloading the page (AJAX). \\% & 2 pts \\
  Fonctional-5 & Implement a feature that improves the application \\% & 2 pts \\
  \hline
  %Total & \multicolumn{2}{r|}{   10 pts \hspace{1 mm}} \\
  \hline
  Bonus & The application is usable by MDE .\\%  & 3 pts \\

  \hline
\end{tabular}


\subsection{Technical Specification (20pts)}

\zero
\begin{tabular}{|p{2cm}|p{10cm}|p{1cm}|}
  \hline
  Constraint & Description \\%  & Points \\

  \hline
  Controller-1 & Develop a controller for the frontend \\% & 1 pts \\
  Controller-3 & At least one controller asks data with a custom SQL query \\% & 1 pts \\

  \hline
  View-1 & Display data (a list) with HTML \\% & 2 pts \\
  View-2 & Display data (an element) with details \\% & 1 pts \\
  View-3 & Create a transition between 2 views \\% & 1 pts \\

  \hline
  Database-1 & Database contains at least one complex entity (that contains a set of other entities) \\% & 3 pts \\

  \hline
  CRUD & At least one complex data entity have create, read, update and delete actions. These actions are located in the backend. \\% & 3 pts \\

  \hline
  AJAX-1 & At least one controller have a method that produce JSON or XML. \\% & 2 pts \\
  AJAX-2 & At least one view gets data dynamically with an AJAX request. \\% & 2 pts \\

  \hline
  Design-1 & Use a CSS framework (Bootstrap\ntm, Zurb Foundation \ntm, PureCSS\ntm, ...). \\% & 1 pts \\
  Design-2 & The application is simple and friendly to use \\% & 2 pts \\
  Design-3 & The application has been developed around prototyping methodology. \ntm \\% & 1 pts \\

  \hline
  %Total & \multicolumn{2}{r|}{   20 pts \hspace{1 mm}} \\

  %\hline
  %Bonus-1 & Integrate the application with members given by LDAP. \\% & 2 pts \\
  %\hline
\end{tabular}

\initnt
\ntt{cf. http://en.wikipedia.org/wiki/Third\_normal\_form}
\ntt{cf. http://getbootstrap.com/}
\ntt{cf. http://foundation.zurb.com/}
\ntt{cf. http://purecss.io/}
\ntt{cf. http://en.wikipedia.org/wiki/Software\_prototyping}


\section{Evaluation}

Evaluation will be made during an oral defense : a jury will check if the students have a good comprehension of the technologies seen during this course; each constraint will be evaluated. The jury will then decide of a mark for the oral defense (10 pts).

\[\text{finalMark} = \frac{\text{technicalMark} + \text{functionalMark} + \text{oralMark}}{2}\]


%\begin{appendices}
%\end{appendices}


\end{document} 